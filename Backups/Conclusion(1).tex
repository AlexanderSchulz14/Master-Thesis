\section{Conclusion}
\label{sec:conclusion}
To summarize, this thesis has investigated how the yield curve and the macroeconomy are related in the United States as well as the Euro Area, especially focusing on the question, if there is a one or bi-directional link. 
By using a two step approach similar to \citet{diebold2006macroeconomy}, first modelling the yield curve using the Nelson-Siegel model to obtain a level, slope and curvature factor and subsequently including said factors in a VAR model containing various other macroeconomic as well as financial variables, the resulting structural shocks have given an indication of said relationship.

A possible extension of this thesis would be to actually use the estimated slope factors in a probabilistic setting like the one proposed by \citet{estrella1991term} and see if said factors, being an approximation of the term spread, are able to correctly predict the onset of past recessions - a finding that would be consistent with the literature.  
Furthermore, testing the performance in yield (curve) forecasting of a general Nelson-Siegel model versus the extended model including macro variables used in this thesis could be further enlightening in regards to the usefulness of a yields-macro model.

%in a machine learning setting 
% various other benchmark models could be further enlightening.  